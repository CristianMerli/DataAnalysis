%%%%%%%%%%%%%%%%%%%%%%%%%%%%%%%%%%%%%%
%         PACKAGES INCLUSION         %                                                                                      % PACKAGES
%%%%%%%%%%%%%%%%%%%%%%%%%%%%%%%%%%%%%%

\documentclass{article}                                                                                                     % Document specs
\usepackage[legalpaper, margin=2cm]{geometry}                                                                               % Document margin
\usepackage[utf8]{inputenc}                                                                                                 % Encoding specs
\usepackage{tocloft}                                                                                                        % Package import for table of contents with dots
\usepackage{hyperref}                                                                                                       % Package import for external references
\usepackage{graphicx}                                                                                                       % Package import to manage images labels and captions
\usepackage{cite}                                                                                                           % Package import for bibliographies (citations)

%%%%%%%%%%%%%%%%%%%%%%%%%%%%%%%%%%%%%%
%           PREAMBLE START           %                                                                                      % PREAMBLE
%%%%%%%%%%%%%%%%%%%%%%%%%%%%%%%%%%%%%%

\setlength{\parindent}{0em}                                                                                                 % Remove indentation at paragraph's start
\hypersetup{colorlinks=true, linkcolor=black, urlcolor=blue}                                                                % External references definition: black links and blue URLs (\href call)
\renewcommand{\contentsname}{Indice}                                                                                        % Change table of contents title
\renewcommand{\cftsecleader}{\cftdotfill{\cftdotsep}}                                                                       % Configure table of contents to display dots
\bibliographystyle{plain}                                                                                                   % Bibliography style (plain standard style)

\title{Analisi dati e calcoli ingegneristici scambiatore di calore \\                                                       % Title definition (printed with \maketitle command)
\large Laboratorio - Gruppo n.5 del 26/11/2021, fisica tecnica [140078] AA 2021/2022}                                       % Subtitle definition (printed with \maketitle command)
\author{Cristian Merli, matr. 211384}                                                                                       % Authors definition (printed with \maketitle command)
\date{07/02/2022}                                                                                                           % Date definition (printed with \maketitle command)

%%%%%%%%%%%%%%%%%%%%%%%%%%%%%%%%%%%%%%
% END OF PREAMBLE and DOCUMENT START %                                                                                      % DOC-START
%%%%%%%%%%%%%%%%%%%%%%%%%%%%%%%%%%%%%%

\begin{document}                                                                                                            % Document-start

\maketitle                                                                                                                  % Plot previously defined title

\vspace{1cm}                                                                                                                % Vertical-space command
  \begin{abstract}                                                                                                          % Abstract creation and following abstract text
    \noindent \textit{Relazione sintetica con lo scopo di descrivere le scelte adottate, e discutere i risultati  ottenuti
    dall'analisi dati e dalle modellazioni ingegneristiche effettuate. Il sistema oggetto di modellazione, è costituito da
    uno scambiatore di calore a fascio tubiero, all'interno del quale scorre acqua in entrambi i circuiti di scambio. Sono
    state effettuate diverse prove, in particolare: due in configurazione equi-corrente e due in configurazione
    contro-corrente, con diversi valori di portata volumetrica di fluido freddo.\newline
    Nei capitoli successivi, verranno elencate le richieste di progetto avanzate dal docente e verranno ripercorse le varie
    tappe, che hanno condotto alla realizzazione e comparazione di diversi modelli ingegneristici più o meno raffinati,
    mediante l'analisi assistita da calcolatore dei dati raccolti durante l'esperienza di laboratorio (effettuata in data
    26/11/2021 con il gruppo numero 5, presso i laboratori del dipartimanto di fisica dell'università di Trento).}
  \end{abstract}                                                                                                            % Abstract end
\vspace{3.5cm}                                                                                                              % Vertical-space command

\vspace{3.5cm}                                                                                                              % Vertical-space command
  \tableofcontents                                                                                                          % Plot table of contents
\pagebreak                                                                                                                  % Go to new page

\section{Richieste}                                                                                                         % Section creation: "Richieste"
\label{sec:project_request}                                                                                                 % "project_request" reference-label definition and following section text
  XSAHxsjhsavxsgavsxgh

\section{Introduzione}                                                                                                      % Section creation: "Introduzione"
\label{sec:introduction}                                                                                                    % "introduction" reference-label definition and following section text
  XSAHxsjhsavxsgavsxgh

\section{Linguaggio di programmazione}                                                                                      % Section creation: "Linguaggio di programmazione"
\label{sec:progr_lang}                                                                                                      % "progr_lang" reference-label definition and following section text
  XSAHxsjhsavxsgavsxgh

\section{Analisi dati}                                                                                                      % Section creation: "Analisi dati"
\label{sec:data_analysis}                                                                                                   % "data_analysis" reference-label definition and following section text
  XSAHxsjhsavxsgavsxgh

\section{Calcoli ingegneristici}                                                                                            % Section creation: "Calcoli ingegneristici"
\label{sec:engineering_calcs}                                                                                               % "engineering_calcs" reference-label definition and following section text
  XSAHxsjhsavxsgavsxgh

\section{Conclusioni}                                                                                                       % Section creation: "conclusioni"
\label{sec:conclusions}                                                                                                     % "conclusions" reference-label definition and following section text
  XSAHxsjhsavxsgavsxgh

\bibliography{biblio/references}                                                                                            % Bibliography inclusion (biblio/references.bib)

\end{document}                                                                                                              % End document code

%%%%%%%%%%%%%%%%%%%%%%%%%%%%%%%%%%%%%%
%            DOCUMENT END            %                                                                                      % DOC-END
%%%%%%%%%%%%%%%%%%%%%%%%%%%%%%%%%%%%%%
