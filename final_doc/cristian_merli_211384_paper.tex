%%%%%%%%%%%%%%%%%%%%%%%%%%%%%%%%%%%%%%
%         PACKAGES INCLUSION         %                                                                                      % PACKAGES
%%%%%%%%%%%%%%%%%%%%%%%%%%%%%%%%%%%%%%

\documentclass{article}                                                                                                     % Document specs
\usepackage[legalpaper, margin=2cm]{geometry}                                                                               % Document margin
\usepackage[utf8]{inputenc}                                                                                                 % Encoding specs
\usepackage{tocloft}                                                                                                        % Package import for table of contents with dots
\usepackage{hyperref}                                                                                                       % Package import for external references
\usepackage{graphicx}                                                                                                       % Package import to manage images labels and captions
\usepackage{cite}                                                                                                           % Package import for bibliographies (citations)

%%%%%%%%%%%%%%%%%%%%%%%%%%%%%%%%%%%%%%
%           PREAMBLE START           %                                                                                      % PREAMBLE
%%%%%%%%%%%%%%%%%%%%%%%%%%%%%%%%%%%%%%

\setlength{\parindent}{0em}                                                                                                 % Remove indentation at paragraph's start
\hypersetup{colorlinks=true, linkcolor=red, urlcolor=blue}                                                                  % External references definition: red links and blue URLs (\href call)
\renewcommand{\cftsecleader}{\cftdotfill{\cftdotsep}}                                                                       % Configure table of contents to display dots
\bibliographystyle{plain}                                                                                                   % Bibliography style (plain standard style)

\title{Dijkstra's algorithm implementation \\                                                                               % Title definition (printed with \maketitle command)
\large C project - G3 n.3, numerical calc and programming [145725] AY 2020/2021}                                            % Subtitle definition (printed with \maketitle command)
\author{Cristian Merli, id. 211384}                                                                                         % Authors definition (printed with \maketitle command)
\date{20/07/2021}                                                                                                           % Date definition (printed with \maketitle command)

%%%%%%%%%%%%%%%%%%%%%%%%%%%%%%%%%%%%%%
% END OF PREAMBLE and DOCUMENT START %                                                                                      % DOC-START
%%%%%%%%%%%%%%%%%%%%%%%%%%%%%%%%%%%%%%

\begin{document}                                                                                                            % Document-start

\maketitle                                                                                                                  % Plot previously defined title

\vspace{1cm}                                                                                                                % Vertical-space command
  \begin{abstract}                                                                                                          % Abstract creation
    \noindent \textit{C-code implementation of Dijkstra's algorithm, inside a dedicated library to manage graphs.           % Abstract text
    This library has also been extended so that a graph's structure could be allocated inside heap to test
    Dijkstras algorithm. With the aim of getting a more user-friendly output, gnuplot takes care of plotting graphics
    to show the structure of the graph and the elaborated shortest path.}
  \end{abstract}                                                                                                            % Abstract end
\vspace{3.5cm}                                                                                                              % Vertical-space command

\vspace{3.5cm}                                                                                                              % Vertical-space command
  \tableofcontents                                                                                                          % Plot table of contents
\pagebreak                                                                                                                  % Go to new page

\section{Project request}                                                                                                   % Section creation: "Project request"
\label{sec:project_request}                                                                                                 % "project_request" reference-label definition
  Dijkstra. Write a software which reads a graph and given two nodes, calculates the minimum path with Dijkstra             % Section text
  algorithm.

\section{Introduction}                                                                                                      % Section creation: "Introduction"
\label{sec:introduction}                                                                                                    % "introduction" reference-label definition
  This document has the main purpose of giving an overview of the project, deeping into theoretical aspects of              % Section text
  Dijkstra's algorithm and how it has been implemented in C-code. While to have further details about technical
  aspects, there is the possibility to consult html documentation of the software (see '\textbf{Doxygen html
  documentation}' section inside '\textbf{README.md}' file).

\section{Dijkstra's algorithm}                                                                                              % Section creation: "Dijkstra's algorithm"
\label{sec:dijkstra_algorithm}                                                                                              % "dijkstra_algorithm" reference-label definition
  Dijkstra's algorithm has been conceived in 1956, by a Dutch computer sientist called Edsger Wybe Dijkstra. The algorithm  % Section text

\bibliography{biblio/references}                                                                                            % Bibliography inclusion (biblio/references.bib)

\end{document}                                                                                                              % End document code

%%%%%%%%%%%%%%%%%%%%%%%%%%%%%%%%%%%%%%
%            DOCUMENT END            %                                                                                      % DOC-END
%%%%%%%%%%%%%%%%%%%%%%%%%%%%%%%%%%%%%%
